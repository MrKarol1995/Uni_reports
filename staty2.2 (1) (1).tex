\documentclass[12pt,letterpaper]{article}
\usepackage[polish,english]{babel} % Para caracteres en español
\usepackage[polish]{babel}
\usepackage{polski}
\usepackage[utf8]{inputenc}	% Para caracteres en español
\usepackage[OT2,OT4]{fontenc}
\usepackage{comment}
\usepackage{amsfonts}

\RequirePackage[numbers]{natbib}
\usepackage[colorlinks=true]{hyperref}

  \hypersetup{
    pdftitle={Symulacje}, %%<--To wymienić 
    pdfauthor={Karol Cieślik},
    colorlinks,
    urlcolor=blue,
    filecolor=magenta,
    citecolor=green, 
    linkbordercolor={1 1 1}, % set to white
    citebordercolor={1 1 1},  % set to white
    urlbordercolor={ 1 1 1}  % set to white
  } 
\RequirePackage[hyperpageref]{backref} 
    \renewcommand*{\backref}[1]{}  
    \renewcommand*{\backrefalt}[4]{
       \ifcase #1 
          No cited.
       \or
          Cited on p. #2.
       \else
          Cited on pp. #2.
       \fi}  

\usepackage{amsmath,amsthm,amsfonts,amscd}%,amssymb}
\usepackage{multirow,booktabs}
\usepackage[table]{xcolor}
\usepackage{fullpage}
\usepackage{lastpage}
\usepackage{enumitem}
\usepackage{fancyhdr}
\usepackage{mathrsfs}
\usepackage{wrapfig}
\usepackage{setspace}
\usepackage{calc}
\usepackage{graphicx}%
\usepackage{mathtools, amsthm, amssymb}
\usepackage{listings}%
\usepackage{multicol}
\usepackage{cancel}
\usepackage[retainorgcmds]{IEEEtrantools}
\usepackage[margin=3cm]{geometry}
\usepackage{floatrow}
\newlength{\tabcont}
\setlength{\parindent}{0.0in}
\setlength{\parskip}{0.05in}
%%Local definition
 \def\bbE{{\mathbb E}}
 \def\bbN{{\mathbb N}}
        \def\cF{{\mathcal F}}
        \def\cH{{\mathcal H}}
        
        \def\R{\Re}
				\def\bD{{\mathbf D}}
				\def\bE{{\mathbf E}}
        \def\bP{{\mathbf P}}
				\def\bT{{\mathbf T}}

        \def\one{{\mathbb I}}
\lstset{ %
    language=Arduino,
    basicstyle=\ttfamily\footnotesize,
    keywordstyle=\color{blue},
    stringstyle=\color{red},
    commentstyle=\color{gray},
    morecomment=[l][\color{magenta}]{\#},
    numbers=left,
    numberstyle=\tiny\color{gray},
    stepnumber=1,
    numbersep=10pt,
    backgroundcolor=\color{lightgray!20},
    frame=single,
    breaklines=true,
    captionpos=b,
}				
\newcommand{\bmalpha}{\boldsymbol \alpha}
\newcommand{\bme}{\mathbf e}
				
\newcommand*{\doi}[1]{\href{http://dx.doi.org/#1}{doi: #1}}
\newcommand*{\MR}[1]{\href{http://www.ams.org/mathscinet-getitem?mr=#1&return=pdf}{MR #1}}
%\newcommand*{\ZBL}[1]{\href{http://www.zentralblatt-math.org/zmath/en/advanced/?q=an:#1&format=complete}{Zbl #1}}

\title{Raport II}
\author{Kinga Heda i Karol Cieślik}
\date{Raport}


\newcommand\course{MST sem. IV}	% <-- nombre del curso
\newcommand\semester{lato 2023/2024}  % <-- semestre
\newcommand\asgnname{2}         % <-- numero o subtítulo de la tarea
\newcommand\yourname{}  % <-- nombre
\newcommand{\vect}[1]{\overline{#1}} % si se quiere cambiar a vector con flecha solo hay que sustituir boldsymbol por vec.
\newcommand{\norm}[1]{\left\lVert#1\right\rVert}	% para denotar la norma euclidiana
\theoremstyle{definition}
\newtheorem{definition}{Definicja}[section]
\newtheorem{reg}{Regla}
\newtheorem{lemma}{Lemat}
\newtheorem{theorem}{Twierdzenie}
\newtheorem{remark}{Komentarz}[section]
\newtheorem{example}{Przykład}[section]


\newtheorem{ejer}{Zadanie}[section]%{EJERCICIO}
\newtheorem{solution}{Odp.}[section]%{Solución}

\pagestyle{fancyplain}
\headheight 32pt
\lhead{\yourname\ \vspace{0.1cm} \\ \course}
\chead{\textbf{\Large Raport Nr. II}}
\rhead{2024/06/19}
\cfoot{Strona \thepage \hspace{1pt} na \pageref{LastPage} \vspace{3mm} \\ \footnotesize \textcolor{gray}{Raport opracowali Kinga Heda i Karol Cieślik kurs \emph{Statystyka stosowana.}}
}
\textheight 580pt
\headsep 10pt
\footskip 40pt
\topmargin = 7pt



\begin{document}
\selectlanguage{polish}
\pagenumbering{roman}
\setcounter{page}{1} %%This command starts the numerations of pages
\maketitle

%\newpage

\tableofcontents

\include{Zadania/Zadania0ASKSz}
%\include{Lecture0Z}

\include{Zadania/Zadania1ASKSz}


\include{Literatura}
\section{Wstęp.}
W niniejszej pracy przedstawimy wyniki analizy statystycznej przeprowadzonej na dwóch próbach pochodzących z populacji  o rozkładzie normalnym(sprawdzono). Analiza ta ma na celu weryfikację różnych hipotez dotyczących parametrów tych rozkładów na poziomie istotności $\alpha = 0.05.$\\

Zweryfikujemy trzy hipotezy dotyczące średniej oraz wariancji. Dla każdej z tych hipotez zostaną narysowane obszary krytyczne oraz obliczone p-wartości. Zwrócimy uwagę na różnice w zależności od dobrania poziomu istotności $\alpha$.\\

Symulacyjne wyznaczymy również prawdopodobieńswto wystąpienia błędów I i II rodzaju później, dla hipotez z zadań 1 i 2, zostaną symulacyjnie wyznaczone prawdopodobieństwa popełnienia błędów pierwszego rodzaju oraz wyznaczymy moc testów, czyli prawdopodobieństwo prawidło- wego odrzucenia fałszywej hipotezy zerowej.\\

Wszystkie analizy zostaną przeprowadzone z użyciem odpowiednich narzędzi statystycznych, a wyniki przedstawione w postaci graficznej oraz tabelarycznej. Naszym celem jest zrozumienie i interpretacja wyników testów statystycznych dla podanych hipotez.
\\

\subsection{Definicje}

\textbf{Hipoteza zerowa} - to twierdzenie, które zakłada brak efektu, różnicy lub związku, ponieważ zakłada stan ”braku zmian” lub ”braku różnicy”. Przykładem może być stwierdzenie, że średnia w populacji wynosi określoną wartość, np: $\mu = 1.5.$\\

\textbf{Hipoteza alternatywna} - jest twierdzeniem przeciwnym do hipotezy zerowej. Zakłada istnienie efektu, różnicy lub związku, który chcemy wykazać. Może to być twierdzenie, że średnia w populacji jest różna od określonej wartości, np: $H_0 : \mu  \neq 1.5$ (alternatywa dwustronna), lub $H_2: \mu > 1.5$ (alternatywa prawostronna) i $H_3: \mu < 1.5$ (alternatywa lewostronna).

\section{Zad. 1}
W pierwszym zadaniu przeprowadzimy analizę danych celem weryfikacji hipotezy zerowej względem trzech hipotez alternatywnych na poziomie istotności $\alpha = 0.05.$ Wiemy, że pobrana próbka X pochodzi z rozkładu normalnego $N(\mu,\sigma)$ o nieznanej średniej $\mu$ i znanym odchyleniu standardowym $\mu = 0.2$. Hipotezę zerową określamy jako:

\begin{equation}
    H_0: \mu = 1.5.
\end{equation} 


Natomiast nasze hipotezy alternatywne to:
\begin{enumerate}
    \centering
    \item $H_1: \mu \neq 1.5,$
    \item $H_2: \mu > 1.5,$
    \item $H_2: \mu < 1.5.$
\end{enumerate}

Najpierw należy przeprowadzić standaryzacje statystyki $\bar X$, pod warunkiem $H_0, Z~N(0,1).$ Definiujemy jako:
\begin{equation}
    Z= \frac{\bar X - \mu_0}{\sigma / \sqrt{n}}
\end{equation} 
\\

Zatem u nas szukana statystyka wynosi $Z=\frac{1.455 - 1.5}{0.2 / 1000}\approx -7.041$
\subsection{Badane obszary}

Rozważamy więc test hipotezy dwustronnej na poziomie istotności $ \alpha = 0.05,$
który określa prawdopodobieństwo popełnienia błedu pierwszego rodzaju, czyli
odrzucenia prawdziwej hipotezy zerowej $H_0$. W przypadku testu dwustronnego,
ta szansa jest równomiernie rozdzielona na dwa ogony rozkładu normalnego,
więc każdy ogon ma poziom istotności $\frac{\alpha}{2}$. Wartości krytyczne $z_\frac{\alpha}{2}$ i $z_1_-_\frac{\alpha}{2}$ to wartości odpowiadające punktom $\alpha$ i $1 -\alpha$ w rozkładzie normalnym standardowym.\\

Zbiór wszystkich wartości statystyki Z, dla których odrzucamy hipotezę zerową nazywamy \textbf{obszarem krytycznym}, który dla hipotezy dwustronnej zdefiniowany jest jako:

\begin{center}
    $C=\{Z:Z < z_\frac{\alpha}{2} \ lub \  Z > z_{1-\frac{\alpha}{2}}\}$
\end{center}

gdzie $Z_1_-_\alpha$ jest oczywiście kwantylem standardowego rozkładu normalnego
rzędu $ 1 -\alpha$. Określiliśmy jego wartość jako $z_1_-_\alpha = z_0_._9_7_5 \approx 1.96.$ A zatem nasz obszar krytyczny możemy zapisać jako:
\begin{center}
    $C=\{Z:Z < -1.96 \ lub \  Z > 1.96\}$
\end{center}

Następnie przeprowadzamy test hipotezy prawostronnej na tym samym poziomie istotności. W przypadku tego testu szansa odrzucenia hipotezy zerowej jest równe szansie, że statystyka \(Z\) należy do prawego ogonu rozkładu normalnego, ograniczonego przez wartość krytyczną \(z_{1 − \alpha} = z_{0.95}\), czyli kwantyl rzędu 0.95 rozkładu standardowego. Jego wartość równa jest \(z_{0.95} \approx 1.645. \)

Obszar krytyczny dla tej statystyki możemy zapisać jako:
\[
C = \{Z : Z > z_1_-_{\alpha}\} = \{Z : Z > z_{0.95}\} = \{Z : Z > 1.645\}
\]

W przypadku testu hiopotezy lewostronnej na poziomie istotności  \(\alpha = 0.05\) badamy czy statystyka \(Z\) znajduje się w lewym ogonie rozkładu normalnego, który ograniczony jest wartość krytyczną \(z_{\alpha} = z_{0.05}\), czyli kwantyl rzędu 0.05 rozkładu standardowego. Wartość tego kwantylu wynosi \(z_{0.05} \approx -1.645. \)

Obszar krytyczny dla tej statystyki możemy zapisać jako:
\[
C = \{Z : Z < z_{\alpha} \} = \{Z : Z < z_{0.05}\} = \{Z : Z < -1.645\}
\]

\subsection{Wizulizacja}
Dzięki zdefiniowanym obszarom krytycznym stwierdzamy, że:
\begin{itemize}
    \item w przypadku hipotezy dwustronnej odrzucamy hipotezę zerową  na rzecz hipotezy alternatywnej, ponieważ \(Z = - 7.041 < - 1.96.\), czyli statystyka \(Z\) zawiera się w obszarze krytycznym,
    \item w przypadku hipotezy prawostronnej odrzucamy hipotezę alternatywną, ponieważ \(Z = - 7.041  \cancel{>}  - 1.96.\), 
    \item hipoteza lewostronna również zastępuje hipotezę zerową jako hipoteza alternatywna, ponieważ \(Z =  -7.041 <
- 1.645.\)
\end{itemize}

Poniższe wykresy przedstawiają interesujace nas zagadnienie:
\begin{figure}[H]
			\centering

				\centering
				\includegraphics[width=\linewidth]{staty1.png}
				\caption{Statystyka Z względem opisanych obszarów krytycznych}
				\label{fig:zdjecie1}
			\hfill
		\end{figure}

\subsection{P-wartości}
Dla badanych hipotez obliczymy p-wartości, czyli najmniejszy poziom istotności \(\alpha\), przy którym rozważana wartość statystyki prowadzi do odrzucenia hipotezy zerowej. Dla hipotezy dwustronnej szukaną wartość znajdujemy obliczając:

\[
p = 2P(Z \geq |z|) = 2(1 - P(Z \leq |z|)) = 2(1 - \phi(|z|))
\]

gdzie \(\phi(x)\) jest dystrybuantą rozkładu normalnego. Obliczając tę statystykę otrzymujemy \(p = 1.902 * 10^{-12}\). Oznacza to, że przy poziomie istotności \(\alpha < p = 1.902 * 10^{-12}\) hipoteza zerowa zostałaby przyjęta.

Dla hipotezy prawostronnej p-wartość otrzymujemy korzystając ze wzoru:

\[
p = P(Z \geq z) = 1 - P(Z \leq z) = 1 - \phi(z) = 0.9999999999990488 \approx 1
\]

oznacza to bardzo małą istotność statystyczną, co sugeruje brak podstaw do odrzucenia hipotezy zerowej przy standardowym poziomie istotności.

P-wartość dla hipotezy lewostronnej otrzymamy korzystając ze wzoru:

\[
p = P(Z \leq z) = \phi(z) \approx 9.512 * 10^{-13}.
\]

Otrzymujemy bardzo niską wartość, co oznacza, że hipoteza alternatywna dla większości przypdków \((\alpha > p)\) hipoteza zerowa zostałaby zastąpiona hipotezą alternatywną.

\subsection{Poziomy istotności}
Powyższa analiza została przeprowadzona na poziomie istotności \(\alpha = 0.05.\) Przyjrzymy się teraz, jaki wpływ ma dobór poziomu istotności na wyniki testowania hipotezy zerowej przeciwko trzem hipotezom alternatywnym. W tym celu porównamy nasze wyniki z wynikami dla zmniejszonego \(\alpha = 0.01\) oraz zwiększonego \(\alpha = 0.1\). Dla każdego z tych przypadków przeprowadzimy analogiczne obliczenia, jak w przykładzie wyżej.

Rozważmy przypadek poziomu istotności \(\alpha = 0.01\). Najpierw znajdujemy kwantyle rzędu 0.995 oraz 0.99.
\[
z_1_-_{\alpha} = z_{0.99}  = - z_{\alpha} = - z_{0.01} = 2.326
\]
\[
z_1_-_\frac{\alpha}{2} = z_{0.995} = - z_\frac{\alpha}{2} = -z_{0.005} = 2.576
\]
Na ich podstawie określamy obszary krytyczne kolejno dla hipotez dwustronnej, prawostronnej i lewostronnej. 
\[
C=\{Z:Z < z_\frac{\alpha}{2} \ lub \  Z > z_{1-\frac{\alpha}{2}}\} = \{Z:Z < -2.576 \ lub \  Z > 2.576\}
\]

\[
C = \{Z : Z > z_1_-_{\alpha}\} = \{Z : Z > 2.326\} 
\]

\[
C = \{Z : Z < z_{\alpha} \} = \{Z : Z < -2.326\}
\]
Ich wizualizajcę oraz porównanie z poziomem istotności $\alpha = 0.05$ przedstawiają poniższe wykresy:

\begin{figure}[H]
			\centering

				\centering
				\includegraphics[width=0.9\linewidth]{staty2.png}
				\caption{Porównanie obszarów krytycznych dla poziomów istotności $\alpha = 0.05$ i $\alpha = 0.01$}
				\label{fig:zdjecie1}
			\hfill
		\end{figure}
Gdy zmniejszamy poziom istotności $\alpha$, obszary krytyczne dla każdej hipotezy stają się mniejsze. W rezultacie, prawdopodobieństwo, że statystyka testowa znajdzie się w obszarze krytycznym, również się zmniejsza. Oznacza to, że rzadziej będziemy odrzucać hipotezę alternatywną, a częściej akceptować hipotezę zerową. Bezpośrednim skutkiem jest zmniejszenie prawdopodobieństwa popełnienia błędu I rodzaju, czyli odrzucenia prawdziwej hipotezy zerowej. Jednakże, w miarę jak maleje prawdopodobieństwo błędu I rodzaju, wzrasta prawdopodobieństwo popełnienia błędu II rodzaju, czyli odrzucenia prawdziwej hipotezy alternatywnej. Wyniki wszystkich przeprowadzonych wcześniej testów pozostają bez zmian.

W przypadku poziomu istotności $\alpha = 0.1$ powtarzamy wszystkie kroki. Zaczynamy od liczenia kwantyli rzędy 0.9 oraz 0.95.
\[
z_1_-_{\alpha} = z_{0.9}  = - z_{\alpha} = - z_{0.1} = 1.282
\]
\[
z_1_-_\frac{\alpha}{2} = z_{0.95} = - z_\frac{\alpha}{2} = -z_{0.05} = 1.645
\]
Pozwala nam to na określenie obszarów krytycznych przy rozważanym poziomie istotnosci dla hipotez kolejno dwustronnej, prawostronnej i lewostronnej.
\[
C=\{Z:Z < z_\frac{\alpha}{2} \ lub \  Z > z_{1-\frac{\alpha}{2}}\} = \{Z:Z < -1.645 \ lub \  Z > 1.645\}
\]

\[
C = \{Z : Z > z_1_-_{\alpha}\} = \{Z : Z > 1.282\} 
\]

\[
C = \{Z : Z < z_{\alpha} \} = \{Z : Z < -1.282\}
\]
Ich wizualizajcę oraz porównanie z poziomem istotności $\alpha = 0.05$ przedstawiają poniższe wykresy:

\begin{figure}[H]
			\centering

				\centering
				\includegraphics[width=0.9\linewidth]{staty3.png}
				\caption{Porównanie obszarów krytycznych dla poziomów istotności $\alpha = 0.05$ i $\alpha = 0.1$}
				\label{fig:zdjecie1}
			\hfill
		\end{figure}

Gdy zwiększamy poziom istotności $\alpha$, obszary krytyczne dla każdej hipotezy rozszerzają się. W efekcie rośnie prawdopodobieństwo, że statystyka testowa znajdzie się w tych obszarach krytycznych. To oznacza, że częściej będziemy odrzucać hipotezę zerową i rzadziej akceptować hipotezę alternatywną. Skutkiem tego jest zwiększenie prawdopodobieństwa popełnienia błędu I rodzaju, czyli odrzucenia prawdziwej hipotezy zerowej. Jednakże, w miarę jak wzrasta prawdopodobieństwo błędu I rodzaju, zmniejsza się prawdopodobieństwo popełnienia błędu II rodzaju, czyli nieodrzucenia prawdziwej hipotezy alternatywnej. Wyniki wszystkich przeprowadzonych wcześniej testów pozostają takie same.

\section{Zad. 2}
W następnym zadaniu wiemy, że pobrana próbka pochodzi z rozkładu normalnego $N(0.2,\sigma^2)$.Tak samo jak w pierwszym zadaiu zweryfikujemy trzy hipotezy. Hipotezę zerową określamy jako:

\begin{equation}
    H_0: \sigma^2 = 1.5
\end{equation} 


Natomiast hipotezy alternatywne to:
\begin{enumerate}
    \centering
    \item $H_1: \sigma^2 \neq 1.5,$
    \item $H_2: \sigma^2 > 1.5,$
    \item $H_2: \sigma^2 < 1.5.$
\end{enumerate}

Rozważmy test dla wariancji w rodzinie rozkładów normalnych. Statystyka:
\begin{center}
    $\chi^2=\frac{(n-1)s^2}{\sigma^2_0}, gdzie \ s^2=\frac{1}{n-1}\sum^n_i_=_1 (X_i - \hat{X})^2$
\end{center}
\\

jest estymatorem nieobciążonym wariancji, przy prawdziwości hipotezy zerowej ma rozkład $\chi^2 \ z \  n - 1$ stopniami swobody. Dla naszych danych mamy:\\
\begin{enumerate}
    \item n=1000
    \item $s^2 \approx 1.6681$
    \item $\sigma^2_0 = 1.5$
\end{enumerate}
\\
A obliczona na podstawie tych wartości statystyka $\chi^2$ wynosi 1108.7487

\subsection{Badane obszary}
Wszystkie przypadki rozpatrujemy na poziomie istotności $\alpha = 0.05$. W przypadku pierwszej hipotezy alternatywnej, gdzie $\sigma ^2 \neq 1.5$ zbiór krytyczny testu ma postać:
\[
C = \left\{x^2 : x^2 \leq \chi^2_{\frac{\alpha}{2}, n-1} \text{ lub } x^2 \geq \chi^2_{1 - \frac{\alpha}{2}, n-1}\right\}
\]
gdzie $x^2 = \frac{(n-1)s^2}{\sigma_0^2}$, a $\chi^2_{\frac{\alpha}{2}$ i $\chi^2_{1 - \frac{\alpha}{2}$ to kwantyle rozkładu $\chi^2$ o $n-1$ stopniach swobody.

Korzystając z wbudowanych bibliotek w python otrzymujemy:
\[
C = (-\infty, 913.3010] \cup [1088.4871, \infty)
\]

Dla hipotezy alternatywnej $\sigma ^2 > 1.5$ zbiór krytyczny możemy wyrazić następująco:
\[
C = \{x^2: x^2 \geq \chi^2_{1 - \alpha, n-1}\}
\]
co daje nam zbiór:
\[
C = [1073.6427, \infty)
\]

Definujemy również zbiór keytyczny dla hipotezy $\sigma ^2 < 1.5$:
\[
C = \{x^2: x^2 \leq \chi^2_{\alpha, n-1}\} = (-\infty, 926.6312]
\]

\subsection{Wizualizacja}
Powyżej zdefiniowane zbiory krytyczne wraz ze statystyką $\chi^2$ przedstawiają poniższe wykresy.
\begin{figure}[H]
			\centering

				\centering
				\includegraphics[width=\linewidth]{staty4.png}
				\caption{Statystyka $\chi^2$ względem opisanych zbiorów krytycznych}
				\label{fig:zdjecie1}
			\hfill
		\end{figure}

\newpage
Dzięki wizualizacji i zdefiniowanym zbiorom krytycznym wnioskujemy, że:
\begin{itemize}
    \item w przypadku hipotezy dwustronnej odrzucamy hipotezę zerową  na rzecz hipotezy alternatywnej, ponieważ  statystyka \(\chi^2\) zawiera się w obszarze krytycznym, przyjmujemy $\sigma^2 \neq 1$,
    \item dla hipotezy prawostronnej również odrzucamy hipotezę zerową i przyjmujemy $\sigma^2 > 1$,
    \item w przyoadku hipotezy lewostronnej statystyka \(\chi^2\) nie mieści się w obszarze krytycznym, czyli przyjmuje hipotezę zerową i $\sigma^2 = 1$.
\end{itemize}

\subsection{P-wartości}
Dla przedstawionych hipotez wyliczymy p-wartości za pomocą wbudowanych bibliotek w python. Hipoteza zerowa zostaje odrzucona na rzecz danej hipotezy alternatywnej, jeśli poziom istotności $\alpha$ jest większy bądź równy uzyskanej p-wartości. Dla hipotezy dwustronnej tę wartość zdefiniowaną mamy jako:
\[
p = 2min(P(\chi^2_n_-_1) \leq x, 1 - P(\chi^2_n_-_1) < x)  
\]
Otrzymujemy $p = 0.015$, czyli hipoteza zerowa zostanie odrzucona dla każdego poziomu istotności $\alpha < 0.015$.

Dla hipotezy prawostronnej otrzymujemy $p = 0.0075$, zgodnie z definicją:
\[
p = 1 - P(\chi^2_n_-_1) < x)
\]

W przypadku hipotezy jednostronnej mamy:
\[
p = P(\chi^2_n_-_1) \leq x = 0.9925
\]

\subsection{Poziomy istotności}
Analizując wyniki na poziomie istotności $\alpha = 0.05$, teraz przeanalizujemy, jak różny wybór poziomu istotności wpływa na rezultaty testowania hipotez. Przeprowadzimy porównanie wyników dla trzech różnych poziomów istotności: zmniejszonego $\alpha = 0.01$, standardowego $\alpha = 0.05$ oraz zwiększonego $\alpha = 0.1$.

Dla każdego z tych przypadków dokonamy analogicznych obliczeń jak przedstawione powyżej, aby lepiej zrozumieć, jak poziom istotności wpływa na nasze decyzje dotyczące odrzucenia lub nie odrzucenia hipotezy zerowej na rzecz hipotez alternatywnych.

Rozważmy przypadek poziomu istotności \(\alpha = 0.01\). Dla znalezienia szukanych obszarów krytycznych potrzebujemy wyznaczyć kwantyle rzędu 0.995, 0.005, 0.99 oraz 0.01,. 
\[
\chi^2_{1 - \frac{\alpha}{2}, n - 1} = \chi^2_{0.995, n - 1} = 1117.8905
\]

\[
\chi^2_{\frac{\alpha}{2}, n - 1} = \chi^2_{0.005, n - 1} = 887.6211
\]

\[
\chi^2_{1 - \alpha, n - 1} = \chi^2_{0.99, n - 1} = 1105.9170
\]

\[
\chi^2_{\alpha, n - 1} = \chi^2_{0.01, n - 1} = 897.9645
\]

Co daje nam zapisać następujące obszary krytyczne odpowiednio dla hipotezy dwustronnej, prawostronnej i lewostronnej:
\[
C = (-\infty, 887.6211]  \cup [1117.8905, \infty)
\]

\[
C = [1105.9170, \infty)
\]

\[
C = (-\infty, 897.9645]
\]

Ich wizualizację przedstawiamy i porównujemy z poziomem istotności $\alpha = 0.05$ na poniższych wykresach:

\begin{figure}[H]
			\centering

				\centering
				\includegraphics[width=\linewidth]{staty5.png}
				\caption{Statystyka $\chi^2$ względem opisanych obszarów krytycznych}
				\label{fig:zdjecie1}
			\hfill
		\end{figure}

Zbadamy również jak zwiększenie poziomu istotności wpłynie na obszary krytyczne. Ponownie wyznaczamy interesujące nas kwantyle rzędu 0.1, 0.9, 0.05 oraz 0.95 i na ich podstawie wyznaczamy zbiory krytyczne dla hipotez kolejno dwustronnej, prawostronnej oraz lewostronnej:
\[
C = (-\infty, 926.6312]  \cup [1073.6427, \infty)
\]

\[
C = [1056.6952, \infty)
\]

\[
C = (-\infty, 942.1612]
\]

W celu zobrazowania różnic przy zwiększonym poziomie istotności wizualizujemy i porównujemy oba na poniższych wykresach:
\begin{figure}[H]
			\centering

				\centering
				\includegraphics[width=\linewidth]{staty6.png}
				\caption{Statystyka $\chi^2$ względem opisanych obszarów krytycznych}
				\label{fig:zdjecie1}
			\hfill
		\end{figure}

Zauważamy, że wraz ze wzrostem poziomu ufności rośnie obszar krytyczny dla wszystkich hipotez, a ze spadkiem poziomu ufności obserwujemy mniejszy obszar krytyczny i maleje prawdopodbieństwo odrzucenia hipotezy zerowej. Dla naszych przypadków zmiana następuje tylko w hipotezie dwustronnej dla poziomu ufności $\alpha = 0.01$, ponieważ w tym przypadku nie odrzucamy hipotezy zerowej na rzecz alternatywnej. W pozostałych przypadkach pozostaje sytucja jak w przypadku $\alpha = 0.05$, mimo, że zmieniają się obszary krytyczne.

\section{Zad. 3}
W ostatnim zadaniu zajmiemy się wyznaczeniem prawdopodobieństwa popełnienia błędów I i II rodzaju dla hipotez z poprzednich zadań. W tym celu wykonamy symulacje, które będą aproksymować te wartości. Wyznaczone prawdopodobieństwa posłużą nam następnie do obliczenia mocy testów.\\

\textbf{Błędem pierwszego rodzaju} nazywamy odrzucenie prawdziwej hipotezy zerowej na rzecz (fałszywej) hipotezy alternatywnej. Prawdopodobieństwo popełnienia tego błędu jest równe wcześniej poznanemu poziomowi istotności $\alpha$.\\

\textbf{Błąd drugiego rodzaju} nazywamy moment odrzucenia prawdziwej hipotezy alternatywnej i przyjęcia fałszywej hipotezy zerowej. Możemy je obliczyć jako $P = 1 - \beta$, gdzie $\beta$ jest zdefinowana jako moc testu. Tym samym moc testu określa także prawdopodobieństwo odrzucenia fałszywej hipotezy zerowej na rzecz prawdziwej hipotezy alternatywnej.\\

\subsection{Błąd pierwszego rodzaju}
 Aby symulacyjnie oszacować błąd pierwszego rodzaju, należy wygenerować próbkę losową z rozkładu normalnego zgodnie z założeniami hipotezy zerowej (średnia $\mu = 1.5$ oraz odchylenie standardowe $\sigma = 0.2$) i sprawdzić, jak często odrzucamy hipotezę zerową.

Przeprowadziliśmy procedurę dla trzech poziomów istotności: $\alpha_1 = 0.05$, $\alpha_2 = 0.01$ i $\alpha_3 = 0.1$, oraz dla trzech różnych hipotez: dwustronnej, prawostronnej, lewostronnej. Wyniki symulacji przedstawiono w tabeli 1.

\newpage
\begin{table}[h!]
    \centering
    \caption{Prawdopodobieństwo błędu I rodzaju}
    \begin{tabular}{cccc}
        \toprule
        Poziom istotności & Hipoteza dwustronna & Hipoteza prawostronna & Hipoteza lewostronna \\
        \midrule
        $\alpha_1 = 0.05$ & 0.051 & 0.048 & 0.054 \\
        $\alpha_2 = 0.01$ & 0.010 & 0.011 & 0.019 \\
        $\alpha_3 = 0.1$ & 0.103 & 0.099 & 0.118 \\
        \bottomrule
    \end{tabular}
\end{table}

Z wyników symulacji wynika, że oszacowane prawdopodobieństwo błędu pierwszego rodzaju jest bardzo zbliżone do teoretycznych wartości dla wszystkich trzech poziomów istotności i rozważanych hipotez. 

\subsection{Błąd drugiego rodzaju}
Aby symulacyjnie oszacować błąd drugiego rodzaju, należy wygenerować próbkę losową z rozkładu normalnego zgodnie z założeniami hipotezy alternatywnej \(H_1\) (ale blisko wartości przyjętych w \(H_0\)) i sprawdzić, jak często przyjmujemy hipotezę zerową.

Przeprowadziliśmy tę procedurę dla jednego poziomu istotności $\alpha_1 = 0.05$ oraz sześciu różnych wartości średniej $\mu$: $\mu_1 = 1.47$, $\mu_2 = 1.48$, $\mu_3 = 1.49$, $\mu_4 = 1.51$, $\mu_5 = 1.52$, $\mu_6 = 1.53$. Wyniki symulacji zostały przedstawione w tabeli 2.

\begin{table}[h!]
    \centering
    \caption{Błąd drugiego rodzaju II}
    \begin{tabular}{cccc}
        \toprule
        $\sigma^2$ & Hipoteza dwustronna & Hipoteza prawostronna & Hipoteza lewostronna \\
        \midrule
        1.47 & 0.919 & - & 0.889 \\
        1.48 & 0.945 & - & 0.909 \\
        1.49 & 0.957 & - & 0.937 \\
        1.51 & 0.945 & 0.924 & - \\
        1.52 & 0.933 & 0.925 & - \\
        1.53 & 0.924 & 0.881 & - \\
        \bottomrule
    \end{tabular}
\end{table}

Widzimy, że błąd drugiego rodzaju jest największy dla wartości bliskich $\sigma^2_0 = 1.5$. Jest to logiczne, ponieważ w takich przypadkach rzeczywista wartość parametru jest bardzo zbliżona do wartości przyjętej w hipotezie zerowej. W konsekwencji istnieje duże prawdopodobieństwo, że hipoteza zerowa zostanie zaakceptowana zamiast hipotezy alternatywnej.


Dla otrzymanych wartości prawdopodobieństwa popełnienia błędu rodzaju
II wyznaczymy także moce testu ze wzoru $\beta = 1 - p$. Wyniki obliczeń
zawarliśmy w tabeli 3.


\begin{table}[h!]
    \centering
    \caption{Moce testu}
    \begin{tabular}{cccc}
        \toprule
        $\mu$ & Hipoteza dwustronna & Hipoteza prawostronna & Hipoteza lewostronna \\
        \midrule
        1.47 & 0.081 & - & 0.111 \\
        1.48 & 0.055 & - & 0.091 \\
        1.49 & 0.043 & - & 0.063 \\
        1.51 & 0.055 & 0.076 & - \\
        1.52 & 0.067 & 0.075 & - \\
        1.53 & 0.076 & 0.119 & - \\
        \bottomrule
    \end{tabular}
\end{table}

\section{Wnioski.}

Wykorzystując wiedzę o testach statystycznych oraz wykorzystując odpowiednie narzędzia informatyczne dokonaliśmy szczegółowej analizy próbek z rozkładu normalnego $N(\mu, \sigma^2)$ pod kątem weryfikacji hipotez alternatywnych.\\

W zadaniu pierwszym przeanalizowaliśmy próbę pochodzącą z populacji normalnej o nieznanej średniej $\mu$ i o znanym odchyleniu standardowym $\sigma = 0.2 $, a następnie zweryfikowaliśmy trzy hipotezy dotyczące średniej populacji.\\
\\

Dla każdej z hipotez wyznaczyliśmy odpowiednie obszary krytyczne oraz obliczono wartości $p$. Wyniki testów statystycznych przedstawiliśmy graficznie, odpowiednio wizualizując obszary krytyczne. Zmiana poziomu istotności $\alpha$ wpływała na szerokość obszarów krytycznych, zwiększając lub zmniejszając prawdopodobieństwo odrzucenia hipotezy zerowej.\\

\\
W zadaniu drugim przeanalizowaliśmy próbę pochodzącą z populacji normalnej o nieznanej wariancji $\sigma^2$ i znanej średniej $\mu= 0.2.$ Na poziomie istotności $\alpha = 0.05$ zweryfikowaliśmy trzy hipotezy dotyczące wariancji populacji.\\

Podobnie jak w zadaniu pierwszym, dla każdej hipotezy wyznaczyliśmy odpowiednie obszary krytyczne oraz obliczyliśmy wartości $p$. Obszary krytyczne były określone przez rozkład $\chi^2$. Analiza wykazała wpływ zmiany poziomu istotności $\alpha$ na obszary krytyczne.\\

W zadaniu trzecim przeprowadziliśmy symulacje szacująca prawdopodobieństw popełnienia błędów I i II rodzaju oraz mocy testów dla hipotez z zadań 1 i 2.\\

Analiza wyników symulacji wykazała, że zwiększenie poziomu istotności $\alpha$ prowadziło do wzrostu prawdopodobieństwa błędu I rodzaju, jednocześnie zmniejszając błąd II rodzaju i zwiększając moc testu.\\

Przeprowadzone analizy i symulacje dostarczyły informacji na temat wpływu poziomu istotności na weryfikację hipotez średniej i wariancji w populacjach normalnych. Graficzne przedstawienie obszarów krytycznych oraz obliczenie wartości $p$ umożliwiło zrozumienie procesu testowania hipotez.\\
Symulacyjne podejście do oszacowania błędów I i II rodzaju oraz mocy testów pozwoliło na ocenę efektywności zastosowanych testów statystycznych.
Zmiana poziomu istotności $\alpha$ ma istotny wpływ na wyniki testów statystycznych, co powinno być uwzględnione przy projektowaniu i interpretacji testów hipotez. Ostateczne wnioski podkreślają znaczenie odpowiedniego doboru poziomu istotności w kontekście specyficznych zastosowań i celów analizy statystycznej.\\



\end{document}

